\par While the relationship between sunspots and cosmicrays is well established,
the exact mechanism of this connection is lightly know compared to their
statistical connection. In this data analysis project the aim was to visualy
and quanitatively represent the cause-affect relationship of sunspots with
cosmic rays.\\
\par To acheive this, data for a specific time interval and region was 
obtained from the NOAA repositories affording a quanitative analysis.\cite{raydata} \cite{spotdata}
Sunspots are tabulated through the use of the Wolf number: $R_z=k(10g+s)$.
Where $k$ is the scaling factoring varying for location and types of intrumentation.
While $g$ representsthe number of sunspot groups and $s$ represents the number
of individual sunspots.\\
\par Using python, each data-set was filtered and
restructed for analysis. The data reflects and indirect correlation between
sunspots and cosmicray. Also, in quanitateing the correlation time between
each corresponding maximum and minimum yielded an average suggesting
a statistical gap between the two occurances.
