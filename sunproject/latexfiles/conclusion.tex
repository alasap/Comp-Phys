Through the use of open-source data and python, a representation of the
relationship between the sun and it's affect seen of earth was created.
Accompanying this, extrema data suggest that sunspot extrema statistically
pre-empt the observed cosmic-ray extrema. Suggesting that sunspots cause
cosmic-ray fluctuations. As described by Porta, 'As is known this time lag
is mainly due to the diffusive and dragging process of cosmic ray particles
in the heliosphere...'\cite{Sierra-Porta2019}. Despite the diffuse
relationship the affects on earth are not just observed in cosmic-rays,
evidently cosmic rays are correlated with cloud cover as well.\cite{Svensmark2016}
Linking sunspots to cloud cover as well.	
