\par Using the files from the NOAA repository the data files were read and 
restructured with python to remove extraneous data.
Sunspots are tabulated through the use of the Wolf number: $R_z=k(10g+s)$.
Where $k$ is the scaling factoring varying for location and types of intrumentation.
While $g$ represents the number of sunspot groups and $s$ represents the number
of individual sunspots.\\

\par After proper structuring
the data was then smoothed through the use of a moving average since the 
original files contain high fluctuations within the data set.
The moving average data-set was utilized in two manners. 
Visual use came through the contruction of a subplot representing
cosmic-rays and sunspots over the course of a $50$ year period. Then was used
numerically to determine the relative maximum's and minimum's within a time
period. After determining the relative extrema the time difference between
their occurances was tabulated and averaged. 
