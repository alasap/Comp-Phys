
In Figure \ref{fig:sundata} the gray spikes correspond to the data-set obtained
from NOAA, while the blue curves represent the moving average of each data-set.
In this figure the data curves reflect an inverse cause-affect relationship between
cosmic-rays and sunspots.\\

\par The correlation strength is further defined by the red and black
dots with each corresponding to the same color below.
Visually these dots still adhere to the correlation, however not all of
the extrema lie directly verticle to one another. Suggesting that this 
relationship is not a direct cause and affect. Table \ref{tbl:timegap} is 
utilized in exploring this connection. The average time-gap between extrema
is about $8.4$ Months. Meaning that sunspot's reach a minimum/maximum about
$255$ days before an extrema is observed in the cosmic-ray data.
