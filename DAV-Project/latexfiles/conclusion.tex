Through the use of open-source data and Python, a representation of the sun and it's affect on earth 
was created.
Accompanying this, extrema data suggest that sunspots statistically
pre-empt the observed cosmic-ray extrema. Suggesting that sunspots cause
cosmic-ray fluctuations. As described by Porta, 'As is known this time lag
is mainly due to the diffusive and dragging process of cosmic ray particles
in the heliosphere...'\cite{Sierra-Porta2019}. Despite this diffuse
relationship the affects on earth are not just observed in cosmic-rays,
evidently cosmic-rays are correlated with cloud cover as well.\cite{Svensmark2016}
Linking not only the sun, but also correlations between sunspots and cosmic-rays
to a diverse amount of observed affects on earth. Further research might include
more comprehensive data-sets, or direct focus to cosmic-ray fluctuations affect's 
on earth's atmosphere.
